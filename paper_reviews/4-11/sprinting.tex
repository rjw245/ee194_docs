%%% Setup %%%
\documentclass{article}
\title{My Title}
\author{Riley Wood}
\begin{document}
\maketitle

%%% The Meat %%%

\section*{Key Ideas} 
As transistors keep shrinking but chip voltage doesn't, power density begins to
increase. This trend has been coined "dark silicon" because it more and more
transistors on chip must remain unpowered or "dark" as they get smaller. This
impacts mobile devices especially which are more difficult to cool. The paper
discusses one way of getting around the utilization wall imposed by power
limitations: exceed the thermal limits only in short bursts to get more work
done per unit time, when needed. They mention interactive applications as their
main target for improvement. These applications generally spend a lot of time
idle, waiting for user input, punctuated by bursts of computation when input is
received. They discuss implications of computational sprinting for thermal,
electrical, and software design.

\section*{Review}
I want to see proof that mobile applications do not demand extended periods of
high performance. That they would benefit from only short bursts.

\section*{Conclusion}
Body of section goes here



%%% References %%%
\bibliographystyle{unsrt}
\bibliography{mybib}

%%% The End %%%
\end{document}
