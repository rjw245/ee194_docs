%%% Setup %%%
\documentclass{article}
\title{Review of ``Razor: a low-power pipeline based on circuit-level timing speculation''~\cite{razor}}
\author{Riley Wood}
\begin{document}
\maketitle

%%% The Meat %%%
\section*{Key Ideas}
The researchers seek to improve the power efficiency of computers by
improving techniques for voltage scaling. Standard techniques are too
conservative and do not lower the voltage past the level needed for the most
power-hungry instructions. The result is that the voltage is higher than it
needs to be the majority of the time. This paper evaluates a method of
dipping the voltage even lower and correcting the occasional errors that
will pop up. They propose having a latch that is clocked a bit late. In the event that
an instruction does not make timing due to the voltage being too low, this latch
will contain the correct result, since it has had more time for the input to
stabilize, and it can be used to restore the correct result to the pipeline.
This requires a one-cycle delay to allow the system to recuperate. The system
keeps track at all times of how frequently errors have been occurring. The
voltage is scaled appropriately so that the number of errors per unit time sits
at an acceptable level.


\section*{Review}
One concern I have when reading their proposal is how they are able to avoid
any sort of ``point of no return'' which would occur when the voltage dips so
low that the Razor latch can no longer make timing. One would hope that
whatever controls they implement on the voltage scaling would scale the
voltage up long before reaching this point, yet they don't talk much about how
tracking and adjusting for pipeline errors would be implemented. In my opinion,
this is a necessary detail to touch upon. The razor latch isn't a believable solution
without a good control scheme to prevent its dangers.


\section*{Conclusions}

%%% References %%%
\bibliographystyle{unsrt}
\bibliography{razor}

%%% The End %%%
\end{document}
