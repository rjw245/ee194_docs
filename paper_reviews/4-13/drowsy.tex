%%% Setup %%%
\documentclass{article}
\usepackage{fullpage}
\title{Review of ``Drowsy Caches: Simple Techniques for Reducing Leakage
Power''~\cite{drowsy}}
\author{Riley Wood}
\begin{document}
\maketitle

%%% The Meat %%%

\section*{Key Ideas}
These researchers seek to lower the power consumption of microprocessors. They
target on-chip caches, which consume a large amount of power, for improvement.
They seek to exploit locality to reduce power consumption by ``dimming'' areas
of the cache that are not being heavily accessed. They claim to reduce the total
energy consumed by caches by 50-75\% with no more than a 1\% decrease in
performance.

\section*{Review}
The claim made in their abstract is very impressive. 50-75\% is a large energy
savings with such a small impact to performance (1\%). But they do not state the
impact this will have on the energy consumption of the entire system. If making
the caches more energy efficient has only a marginal effect on the overall
energy consumption of the computer, then the work is not very useful. They
evaluate their design thoroughly by testing many design corners, down
to whether or not to put tags into drowsy mode along with data. I was also
impressed at the degree to which they evaluated potential weaknesses in their
novel designs, such as the DVS drowsy memory cell schematic. They saw the
potential for memory corruption due to crosstalk, tested if their design was
susceptible, and presented the results that prove it is not. This is very
convincing.

\section*{Conclusion}
Given the degree to which they tested their designs and explored every design
corner, it's clear that a lot of thought was put into this project. My main
concern remains that I haven't been convinced that making caches more power
efficient will significantly benefit the whole system. While my searches online
reinforce the idea that caches consume a large fraction of the total chip power,
I would like to see measurements that make this clear. But overall, I agree with
the approach this paper takes to reducing power consumption. At the time, DVFS
was beginning to be applied to processor cores. It makes a lot of sense to extend
this technique (at least the voltage scaling) to caches.

%%% References %%%
\bibliographystyle{unsrt}
\bibliography{drowsy}

%%% The End %%%
\end{document}
