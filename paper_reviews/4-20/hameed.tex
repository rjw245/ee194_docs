%%% Setup %%%
\documentclass{article}
\usepackage{fullpage}
\title{Review of ``Understanding Sources of Inefficiency in General-Purpose
Chips'' \cite{review}}
\author{Riley Wood}
\begin{document}
\maketitle

%%% The Meat %%%
Using three late days!

\section*{Key Ideas}
The paper seeks to explore how specialization can benefit a chip multiprocessor
(CMP) in terms of performance and power consumption. It is generally known that a
specially-designed ASIC will outperform a CMP in these areas, but the paper aims
to point out exactly why, in an effort to appropriate the most beneficial
characteristics for new CMP designs. They use H.264 video encoding as their
primary application for demonstration. H.264 video encoding is something which
is typically carried out by an ASIC rather than a general-purpose processor
because the ASIC ends up being much better in all respects (power, area, and
performance). They begin with an implementation of H.264 on a general processor
that lags behind the ASIC drastically. They gradually apply improvements to the
CMP which incorporate aspects of the ASIC to improve performance and
resource consumption of the CMP. They conclude that reducing the energy cost of
CMPs is difficult, as energy overhead will tend to dominate.

\section*{Review}
I believe they selected an appropriate application for testing, since H.264
video encoding is today widely implemented using hardware acceleration. However,
the fact that industry has chosen hardware acceleration over specialized
processors means that industry trends have not aligned with the paper's
conclusion: that specialization of the processor itself is the best way to
maximize performance given energy constraints. I don't think the paper addresses
the hardware acceleration school of thought well enough. In the introduction,
the authors mention that work is being done to make generating customized
hardware easier. They never weigh this option against augmenting the processor.
In fact, in their conclusion, they admit that extending the processor is
difficult and additional tools will be needed if this is to be made easy for
designers. So this approach is no easier than custom designing hardware.

\section*{Conclusion}
This paper offers a lot of insight into the ways in which ASICs outperform
processors in ways that are very relevant to today's trend toward power
efficiency. Unfortunately, I disagree with their claim that extending the
processor with specialized functionality is the best approach to averting the
utilization wall. I am not convinced that this beats integrating what are
effectively ASICs into the chip die as accelerators.

%%% References %%%
\bibliographystyle{unsrt}
\bibliography{hameed}

%%% The End %%%
\end{document}
